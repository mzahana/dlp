%===============================================================================
% $Id: ifacconf.tex 19 2011-10-27 09:32:13Z jpuente $  
% Template for IFAC meeting papers
% Copyright (c) 2007-2008 International Federation of Automatic Control
%===============================================================================
\documentclass{article}

\usepackage{graphicx}      % include this line if your document contains figures
\usepackage{natbib}        % required for bibliography
\usepackage{amsmath}
%\usepackage{amstex}
\usepackage{amssymb}

% url
\usepackage[hyphenbreaks]{breakurl}
\usepackage[hyphens]{url}

% for algorithms
\usepackage{algorithm}
\usepackage{algorithmicx,algpseudocode}

% for tables
\usepackage{array}
\usepackage{tabu}

% use for multi-figures in one figure
\usepackage{subcaption}

% for caption formatting
\usepackage{caption}

% for colored text
\usepackage{color}
%===============================================================================
\title{Implementation Notes for the Distributed LP Framework}
\author{Mohamed Abdlekader}
\date{April 2017}
 
\begin{document}
 
\begin{titlepage}
\maketitle
\end{titlepage}

%===============================================================================

\section{Introduction}
This documents describes some implementation details of the distributed LP framework. This document uses the corresponding \textit{IFAC2017} paper as a reference.

\section{Dynamics}

\begin{equation}
\mathbf{x}^+ = \mathbf{x} + ( \mathbf{B}_{\text{in}} - \mathbf{B}_{\text{out}})\mathbf{u}
\end{equation}

Also,
\[
\mathbf{B} = \mathbf{B}_{\text{in}} - \mathbf{B}_{\text{out}} \in \mathcal{R}^{n_s \times n_u}
\]
Where $\mathbf{u}$ is
\begin{equation*}
\mathbf{u} = [\mathbf{u}_{s_1}^T ~\mathbf{u}_{s_2}^T\dots  \mathbf{u}_{s_{n_s}}^T]^T \in {\cal R}^{n_u}. 
\end{equation*}
and $\mathbf{u}_{s_i}$ is
\begin{equation}
\label{sectorinput}
\mathbf{u}_{s_i}= [ u_{s_i \rightarrow s_1} \dots u_{s_i \rightarrow s_{i-1}} {\color{red}u_{s_i \rightarrow s_{i}} } u_{s_i \rightarrow s_{i+1}} \dots\\ u_{s_i \rightarrow s_{n_s}}]^T .
\end{equation}
Notice that Eq. \ref{sectorinput} is different from the definition in the paper. That is because ${\color{red} u_{s_i \rightarrow s_{i}}}$ is added to the $\mathbf{u}_{s_i}$ vector. \textit{However, it is always multiplied by zero}. It is added in order to make the implementation easier later on.
Therefore, ${\color{red}n_u = n_s^2}$ in this implementation.

\paragraph{Dynamics constraints}: dynamics over prediction time horizon $T_p$,
\begin{equation}
\label{dynamicsovertp}
\mathbf{X}= \mathbf{T}_u \mathbf{U} + \mathbf{T}_{x_0}\mathbf{x}[0]
\end{equation}

where $\mathbf{T}_u$ is,
\[
\mathbf{T}_u=
  \begin{bmatrix}
    \mathbf{B} & \cdots & \mathbf{0} \\
    \mathbf{B} & \mathbf{B} & \cdots & \mathbf{0}\\
    \cdots & \vdots & \cdots\\
    \mathbf{B} & \mathbf{B} & \cdots & \mathbf{B}
  \end{bmatrix}_{(n_s T_p) \times (n_u T_p)}
\]
and,
\[
\mathbf{T}_{x_0}\mathbf{x}[0]=
\begin{bmatrix}
\mathbf{x}[0]\\
\vdots \\
\mathbf{x}[0]
\end{bmatrix}_{n_s T_p \times 1}
\]
Assuming that the optimization vector is,
\begin{equation}
\label{optimizationvector}
\hat{\mathbf{X}} = 
\begin{bmatrix}
\mathbf{X} \\
\mathbf{U}
\end{bmatrix}_{(n_s+n_u)T_p \times 1}
\end{equation}
Eq \ref{dynamicsovertp} can be written as, 

\begin{equation}
\begin{bmatrix}
\mathbf{I}_{n_s T_p} & -\mathbf{T}_u
\end{bmatrix}
\hat{\mathbf{X}} = \mathbf{T}_{x_0}\mathbf{x}[0]
\end{equation}
It can be converted to inequalities as follows,
\begin{equation}
\label{comapctdynamicsconstraints}
\begin{bmatrix}
\mathbf{I}_{n_s T_p} & -\mathbf{T}_u\\
-\mathbf{I}_{n_s T_p} & \mathbf{T}_u
\end{bmatrix}_{2n_s T_p \times (n_s+n_u)T_p}
\hat{\mathbf{X}}
\leq
\begin{bmatrix}
\mathbf{T}_{x_0}\mathbf{x}[0]\\
-\mathbf{T}_{x_0}\mathbf{x}[0]
\end{bmatrix}_{2n_s T_p \times 1}
\end{equation}
Which can be re-written as 
\[
\mathbf{A}_{\text{dynamics}} \hat{\mathbf{X}} \leq \mathbf{b}_{\text{dynamics}}
\]
\section{Flow Constraints}
Flow constraints over prediction time horizon are described by,
\begin{equation}
\label{flowconstraints}
\mathbf{T}_{u,c} \mathbf{U} \leq \mathbf{T}_{x_0,c} \mathbf{x}[0]
\end{equation}
where,
\[
\mathbf{T}_{u,c} =
\begin{bmatrix}
\mathbf{B}_{\text{out}} & \mathbf{0} & \cdots & \cdots & \mathbf{0}\\
- \mathbf{B} & \mathbf{B}_{\text{out}} & \mathbf{0} & \cdots & \mathbf{0}\\
- \mathbf{B} & -\mathbf{B} & \mathbf{B}_{\text{out}} & \cdots & \mathbf{0}\\
\vdots & \vdots & \vdots & \vdots & \vdots\\
- \mathbf{B} & - \mathbf{B} & - \mathbf{B} & - \mathbf{B} & \mathbf{B}_{\text{out}}

\end{bmatrix}
\]

$\mathbf{T}_{u,c} \in \mathcal{R}^{n_s T_p \times n_u T_p}$, and 
\[
\mathbf{T}_{x_0,c} \mathbf{x}[0] =
\begin{bmatrix}
\mathbf{x}[0]\\
\vdots \\
\mathbf{x}[0]
\end{bmatrix}_{n_s T_p \times 1}
\]
Using the same optimization vector in (\ref{optimizationvector}), constraints (\ref{flowconstraints}) can be written as,
\begin{equation}
\label{compactflowconstraints}
\begin{bmatrix}
\mathbf{0}_{n_s T_p \times n_s T_p} & \mathbf{T}_{u,c}
\end{bmatrix}_{n_s T_p \times (n_s+n_u)T_p}
\hat{\mathbf{X}} \leq  \mathbf{T}_{x_0,c} \mathbf{x}[0]
\end{equation}
which can be written as,
\[
\mathbf{A}_{\text{flow}} \hat{\mathbf{X}} \leq \mathbf{b}_{\text{flow}}
\]

\section{Boundary Constraints}
Constraints on optimization vector $\hat{\mathbf{X}}$ are,
\[
\mathbf{0} \leq \hat{\mathbf{X}} \leq \mathbf{1}
\]
or,

\begin{align*}
\hat{\mathbf{X}} \leq \mathbf{1}_{(n_s+n_u)T_p} \\
-\hat{\mathbf{X}} \leq \mathbf{0}_{(n_s+n_u)T_p}
\end{align*}

which can be written as,
\begin{equation}
\label{boundariesconstraints}
\begin{bmatrix}
\mathbf{I}\\
-\mathbf{I}
\end{bmatrix}_{2 (n_s+n_u)T_p \times (n_s+n_u)T_p}
\hat{\mathbf{X}} \leq
\begin{bmatrix}
\mathbf{1}\\ \mathbf{0}
\end{bmatrix}
\end{equation}
which can be written as
\[
\mathbf{A}_{\text{boundary}} \hat{\mathbf{X}} \leq \mathbf{b}_{\text{boundary}}
\]

\section{Compact LP Form}
Using inequality (\ref{comapctdynamicsconstraints}), (\ref{compactflowconstraints}), and (\ref{boundariesconstraints}), the LP problem can be written in compact from as,
\begin{equation}
\begin{aligned}
&\min_{\hat{\mathbf{X}}} \mathbf{C}^T \hat{\mathbf{X}} \\
\text{s.t.} \;  \\
&
\begin{bmatrix}
\mathbf{I}_{n_s T_p} & -\mathbf{T}_u\\
-\mathbf{I}_{n_s T_p} & \mathbf{T}_u\\
\mathbf{0}_{n_s T_p \times n_s T_p} & \mathbf{T}_{u,c} \\
\mathbf{I}_{(n_s+n_u)T_p}\\
-\mathbf{I}_{(n_s+n_u)T_p}\\
\end{bmatrix}_{(5n_s+2n_u)Tp \times (n_s+n_u)T_p} \hat{\mathbf{X}}
\leq
\begin{bmatrix}
\mathbf{T}_{x_0}\mathbf{x}[0]\\
-\mathbf{T}_{x_0}\mathbf{x}[0]\\
\mathbf{T}_{x_0,c} \mathbf{x}[0]\\
\mathbf{1}_{(n_s+n_u)T_p}\\
\mathbf{0}_{(n_s+n_u)T_p}
\end{bmatrix}_{(5n_s+2n_u)Tp \times 1}
\end{aligned}
\end{equation}
which can be written as,
\begin{equation}
\begin{bmatrix}
\mathbf{A}_{\text{dynamics}}\\
\mathbf{A}_{\text{flow}}\\
\mathbf{A}_{\text{boundary}}
\end{bmatrix}
\hat{\mathbf{X}}
\leq 
\begin{bmatrix}
\mathbf{b}_{\text{dynamics}}\\
\mathbf{b}_{\text{flow}}\\
\mathbf{b}_{\text{boundary}}
\end{bmatrix}
\end{equation}
%\bibliography{IFAC_Dist.bib}

\end{document}
